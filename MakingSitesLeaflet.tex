\documentclass[letterpage,foldmark,10pt]{leaflet}
\usepackage{lettrine}
\usepackage{minted}
\usepackage{framed}
\usepackage{hyperref}
\title{Handout for Code Your Own Site}
\author{}
\date{}
\pagenumbering{gobble}

\begin{document}
\maketitle
\section{Getting Started}
  Welcome to the class!

  Copies of the slides with notes are stored at:

  \url{https://github.com/clarissalittler/multcolib-lectures/blob/master/CodeYourOwnAlt.pdf}

  From here you can click ``Download'' and read the slides any time.
\section{Links to resources}
\begin{itemize}
  \item General JavaScript notes: \url{http://bit.ly/2ahKOZc}
  \item Tutorial on HTML and CSS: \url{http://bit.ly/2af3ifd}
  \item Tutorial on JavaScript and the Document Object Model: \url{http://bit.ly/2abXKFa}
  \item Neocities: \url{https://neocities.org/}
\end{itemize}

\section{Glossary of Terms}
\begin{description}
  \item [HTML] HyperText Markup Language, the content of a web page
  \item [CSS] Cascading Style Sheets, how pages look
  \item [JavaScript] A programming language that runs in the browser and provides interaction
  \item [DOM] Document Object Model, which connects the JavaScript code to the page
\end{description}

\section{Common Tags}
\begin{description}
  \item [h1] heading tag, for headlines, section headings, chapter titles
  \item [p] paragraph tag, holds basic text
  \item [ol] ordered list tag, for numbered or enumerated lists
  \item [ul] unordered list tag, for bulleted---unnumbered---lists
  \item [li] list items, go inside ol or li tags
  \item [style] CSS code goes between the style tags
  \item [script] JavaScript code goes between the script tags
  \item [a] anchor tag is used to make links
\end{description}

\section{HTML Template}
\begin{minted}{html}
  <!doctype html>
  <html>
    <head>
    </head>
    <body>
    </body>
  </html>
\end{minted}
\newpage
\section{HTML Example}
\begin{minted}{html}
  <!doctype html>
  <html>
    <body>
      <h1>This is a heading</h1>
      <p>This is a paragraph</p>
    </body>
  </html>
\end{minted}

\section{Basic Selectors}
\begin{description}
  \item [.name] selection by class
  \item [\#name] selection by ID
  \item [name] selection by tag name
\end{description}

\section{Common CSS Properties}
\begin{description}
  \item [width] the width of the element
  \item [height] the height of the element
  \item [color] the color of the text
  \item [background-color] the color of the background of the element
  \item [display] how the element is displayed: block and none are two possible values
\end{description}
\section{CSS Example}
\begin{minted}{html}
  <!doctype html>
  <html>
    <head>
      <style>
        p {
          color: red;
        }
      </style>
    </head>
    <body>
      <h1>This is a heading</h1>
      <p>This is a paragraph</p>
    </body>
  </html>
\end{minted}
\section{Basic JavaScript Syntax}
\subsection{Creating Variables}
\begin{minted}{js}
  var variableName = 20;
  variableName;
  variableName = 30;
\end{minted}
\subsection{Arithmetic}
\begin{minted}{js}
  10 + 10;
  20 - 5;
  30 * 3;
  30 / 10;
\end{minted}
\subsection{Strings}
\begin{minted}{js}
  "this is a string";
  'as is this';
  "as is 'this'";
  'and is "this"';
\end{minted}
\begin{framed}
  You can mix quotation types for typesetting purposes, but otherwise they're the same.
\end{framed}
\subsection{Functions}
\subsubsection{Creating functions}
\begin{minted}{js}
  function functionName (x) {
    console.log(x);
    return x + 10;
  }
\end{minted}
\subsubsection{Using functions}
\begin{minted}{js}
  functionName(10);
  console.log("thing");
\end{minted}
\subsection{Objects}
\begin{minted}{js}
  var myObject = {property1 : value1, 
                  property2 : value2};
  myObject.property1;
  myObject.property2 = 100;
\end{minted}
\section{Document Object Model}
\subsection{Preliminary}
\begin{minted}{js}
  window.onload = function () {
    (your code here)
  };
\end{minted}
\subsection{Objects and functions}
\begin{description}
  \item [document] the main object that connects JavaScript to the web page
  \item [document.createElement] function that takes a tag name and returns an object
  \item [document.findElementById] function that retrieves an element by ID
  \item [document.createTextNode] function that creates text from strings
  \item [e.appendChild] function that attaches one element to another
  \item [e.style] object that contains CSS style properties
  \item [e.addClass] function that adds a CSS class to an element
\end{description}

\end{document}